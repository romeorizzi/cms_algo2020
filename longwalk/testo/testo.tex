
\documentclass[a4paper,11pt]{article}

\usepackage[utf8x]{inputenc}
\SetUnicodeOption{mathletters}
\SetUnicodeOption{autogenerated}

\usepackage[italian]{babel}
\usepackage{booktabs}
\usepackage{mathpazo}
\usepackage{graphicx}
\usepackage[left=2cm, right=2cm, bottom=3cm]{geometry}
\frenchspacing

\begin{document}

\noindent {\Huge longwalk (\texttt{longwalk})}


\vspace{0.5cm}
\noindent {\Large (tempo limite 1 sec)}

\section*{Descrizione del problema}
  
      In un torneo a $N$ giocatori si disputano $M$ partite.
      Ciascuna partita si svolge tra due giocatori $p$ e $q$,
      con $1 ≤ p,q ≤ N $, e non si verifica alcun pareggio tra i
      giocatori. Tuttavia, possono esserci più partite tra gli stessi giocatori,
      con esiti indipendenti dalle partite precedenti.
    
      Siamo interessati a scoprire se esista una sequenza di giocatori $p_{0}, \ldots, p_{k-1}$ tale che il giocatore $p_i$ abbia vinto in almeno una partita contro il giocatore $p_{i+1 .mod. k}$ per ogni $i=0,\ldots, k-1$.
      Qualora questa non esista,
      si richiede di determinare la pi\`u lunga sequenza $p_{1}, \ldots, p_{\overline{k}}$
      di giocatori tale che ogni giocatore della sequenza
      tranne l'ultimo abbia vinto in almeno una partita contro il giocatore
      successivo. 
    

\section*{Dati di input}
  
      Il file di input è composto da $M+1$ righe.
    
      La prima riga contiene due numeri interi $N$ e $M$,
      rispettivamente il numero di giocatori e il numero di partite.
    
      Ciascuna delle $M$ righe successive contiene due numeri interi
      $p$ e $q$ e rappresenta una singola partita tra 
      $p$ e $q$ in cui $p$ è il vincitore.
    

\section*{Dati di output}
  
      Il file di output è composto da 2 righe.
    
      Qualora esista una sequenza $p_{0}, \ldots, p_{k-1}$ tale che 
      ciascun giocatore a parte l'ultimo ha vinto in almeno una partita contro il giocatore successivo, e l'ultimo giocatore ha vinto contro il primo, la prima riga deve contenere l'intero $-1$ e l'intero $k$.
      La seconda riga deve contenere $k$ numeri interi $p_{0}, \ldots, p_{k-1}$
      che rappresentano una sequenza di giocatori con queste caratteristiche.
    
      Altrimenti, la prima riga deve contenere
      la massima lunghezza $\overline{k}$ di una sequenza in cui ogni giocatore tranne l'ultimo abbia vinto una qualche partita contro il successivo;
      la seconda riga deve contenere $\overline{k}$ numeri interi $p_{1}, \ldots, p_{\overline{k}}$
      che rappresentino una tale sequenza.
    
  \section*{Assunzioni}
  \begin{itemize}
  
    \item $2 ≤ N ≤  100000$
    \item $2 ≤ M ≤ 1000000$
  \end{itemize}

\section*{Esempi di input/output}

  
    \noindent
    \begin{tabular}{p{11cm}|p{5cm}}
    \toprule
    \textbf{File \texttt{input.txt}}
    & \textbf{File \texttt{output.txt}}
    \\
    \midrule
    \scriptsize
    \begin{verbatim}
5 9
1 4
4 2
5 4
1 4
1 4
4 2
4 2
5 4
1 5
      \end{verbatim}
    &
    \scriptsize
    \begin{verbatim}
4
1 5 4 2
      \end{verbatim}
    \\
    \bottomrule
    \end{tabular}
  
    \noindent
    \begin{tabular}{p{11cm}|p{5cm}}
    \toprule
    \textbf{File \texttt{input.txt}}
    & \textbf{File \texttt{output.txt}}
    \\
    \midrule
    \scriptsize
    \begin{verbatim}
5 11
5 4
5 3
2 3
4 3
2 5
1 2
4 2
1 2
4 3
1 2
2 3
      \end{verbatim}
    &
    \scriptsize
    \begin{verbatim}
−1 3
2 5 4
      \end{verbatim}
    \\
    \bottomrule
    \end{tabular}
  


\end{document}
