% Template per generare 

\documentclass[a4paper,11pt]{article}
\usepackage{lmodern}
\renewcommand*\familydefault{\sfdefault}
\usepackage[utf8]{inputenc}
\usepackage[T1]{fontenc}
\usepackage[italian]{babel}
\usepackage{indentfirst}
\usepackage{graphicx}
\usepackage{tikz}
\newcommand*\circled[1]{\tikz[baseline=(char.base)]{
		\node[shape=circle,draw,inner sep=2pt] (char) {#1};}}
% \usepackage[group-separator={\,}]{siunitx}
\usepackage[left=2cm, right=2cm, bottom=3cm]{geometry}
\frenchspacing

\newcommand{\num}[1]{#1}

% Macro varie...
\newcommand{\file}[1]{\texttt{#1}}
\renewcommand{\arraystretch}{1.3}
\newcommand{\esempio}[2]{
\noindent\begin{minipage}{\textwidth}
\begin{tabular}{|p{11cm}|p{5cm}|}
	\hline
	\textbf{File \file{input.txt}} & \textbf{File \file{output.txt}}\\
	\hline
	\tt \small #1 &
	\tt \small #2 \\
	\hline
\end{tabular}
\end{minipage}
}

% Dati del task
\newcommand{\gara}{phd2015}
\newcommand{\nome}{Numero di piastrellature di bagno 1xn}
\newcommand{\nomebreve}{piastrelle\_count}

\begin{document}
% Intestazione
\noindent{\Large \gara}
\vspace{0.5cm}

\noindent{\Huge \textbf \nome~(\texttt{\nomebreve})}

% Descrizione del task
\section*{Descrizione del problema}
Pippo ha un corridoio di dimensione $1 \times N$ da piastellare utilizzando solo piastrelle quadrate di dimensioni $1 \times 1$ e piastrelle rettangolari di dimensioni $1 \times 2$.

Scrivete un programma che, dato $N$, restituisca il numero di piastrallature possibili.

% Input
\section*{Input (da stdin)}
Il vostro programma riceve in input l'intero $N$, la lunghezza del corridoio da piastrellare.

% Output
\section*{Output (su stdout)}
Restituite in output un unico intero: il numero di tutte le diverse piastrallature possibili di un corridoio di lunghezza $N$.

% Assunzioni
\section*{Assunzioni sulle istanze da punti}

\begin{itemize}
\item $1 \le N \le 32$
\item L’output sarà abbastanza piccolo da poter essere mantenuto dentro un `long long int`
\end{itemize}

% Subtasks
\section*{Subtask}
\begin{itemize}
\item \textbf{Subtask 1 [25 punti]:} $N \leq 10$
\item \textbf{Subtask 2 [25 punti]:} $N \leq 20$
\item \textbf{Subtask 3 [25 punti]:} $N \leq 30$
\item \textbf{Subtask 4 [25 punti]:} $N \leq 32$
\item \textbf{Subtask 5 [0 punti]:} nell'ordine, le istanze $N = 0$, $N = 33$, $N = 34$, $N = 35$
\end{itemize}

% Esempi
\section*{Esempi di input/output}

\esempio{3}{3}

\esempio{4}{5}

\end{document}
