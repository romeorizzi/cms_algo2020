
\documentclass[a4paper,11pt]{article}

\usepackage[utf8x]{inputenc}
\SetUnicodeOption{mathletters}
\SetUnicodeOption{autogenerated}

\usepackage[italian]{babel}
\usepackage{booktabs}
\usepackage{mathpazo}
\usepackage{graphicx}
\usepackage[left=2cm, right=2cm, bottom=3cm]{geometry}
\frenchspacing

\begin{document}
\noindent {\Large Finali Nazionali 2004}
\vspace{0.5cm}

\noindent {\Huge Arcobaleno e Collage (\texttt{collage})}


\section*{Descrizione del problema}
  Nel pianeta Wobniar ogni mattina splende un bellissimo e caratteristico
arcobaleno. La particolarità consiste nella disposizione dei colori, che
possono presentarsi più volte all'interno dell'arco.Il famoso artista Ed Esor decide un giorno di voler catturare lo
splendore dell'arco in un collage di strisce colorate.
I suoi quadri sono valutati dalla critica in modo assai curioso. Tra
questi, i critici giudicano come migliori quelli
costituiti da poche diverse strisce: per
l'intellighenzia di Wobniar, usare poche strisce colorate è una tecnica
sopraffina che fa della sovrapposizione un'arte.Aiuta Ed Esor a minimizzare il numero di strisce del suo collage!Se, ad esempio, l'arcobaleno fosse composto da sole 3 strisce di 2
colori diversi alternati, Ed Esor riuscirebbe a fare un collage usando
due sole strisce di carta: una, disposta come base, dello stesso
colore delle due strisce alle estremità dell'arcobaleno, l'altra
posata sul centro della prima.

\section*{Dati di input}
  Il file \texttt{input.txt} è composto da
più numeri interi: sulla prima riga il numero $N$ a indicare il
numero di strisce dell'arcobaleno; sulla seconda riga sono posti $N$
numeri interi $C_{1}$, $C_{2}$, ..., $C_{N}$
a indicare i colori della striscia.
Ogni colore $C_{i}$ è un numero
intero compreso tra 0 e 255.
Strisce uniformi di colore sono indicate da più numeri uguali
consecutivi.

\section*{Dati di output}
  Il file \texttt{output.txt} dovrà contenere un
unico numero: il numero minimo di strisce per riprodurre l'arcobaleno.
  \section*{Assunzioni}
  \begin{itemize}
  
    \item $0 < N ≤ 1000$
    \item Per ogni $C_{i}$, $0 ≤
C_{i}$
    \item Il tempo di esecuzione massimo è fissato in 4 secondi
  \end{itemize}

\section*{Esempi di input/output}

  
    \noindent
    \begin{tabular}{p{11cm}|p{5cm}}
    \toprule
    \textbf{File \texttt{input.txt}}
    & \textbf{File \texttt{output.txt}}
    \\
    \midrule
    \scriptsize
    \begin{verbatim}
3 
1 2 1
\end{verbatim}
    &
    \scriptsize
    \begin{verbatim}
2
\end{verbatim}
    \\
    \bottomrule
    \end{tabular}
  
    \noindent
    \begin{tabular}{p{11cm}|p{5cm}}
    \toprule
    \textbf{File \texttt{input.txt}}
    & \textbf{File \texttt{output.txt}}
    \\
    \midrule
    \scriptsize
    \begin{verbatim}
7
1 1 2 3 1 2 1
\end{verbatim}
    &
    \scriptsize
    \begin{verbatim}
4
\end{verbatim}
    \\
    \bottomrule
    \end{tabular}
  


\end{document}
