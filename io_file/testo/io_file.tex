\renewcommand{\nomebreve}{io\_file}
\renewcommand{\titolo}{Gestire {\bf i}nput ed {\bf o}utput da file}

\mbox{\ }
\vspace{-0.6cm}

\introduzione{}

Ricevete in input un numero intero e,
in output, dovete ritornare lo stesso identico numero. Nessuna elaborazione \`e richiesta (\`e un esercizio di test ed acclimatamento).

Mi spiego meglio:
nella stessa cartella in cui il vostro programma verr\`a messo in esecuzione
si trova un file di testo di nome \texttt{input.txt}.
In questo file \`e scritto solamente un intero.
Voi dovete creare un file di testo di nome \texttt{output.txt}
e ricopiarci dentro lo stesso identico valore intero.
In pratica, col file \texttt{output.txt},
creerete una copia del file \texttt{input.txt}.

   
% Esempi
\sezionetesto{Esempio di input/output}
\esempio{17}{17}

% Assunzioni
\sezionetesto{Assunzioni}
\begin{itemize}[nolistsep, noitemsep]
  \item il numero intero in questione ha valore assoluto al massimo $1000$,
        pu\`o quindi essere convenientemete memorizzato in una variabile \texttt{int} del C o c++, oppure in una variabile di tipo \texttt{integer} del Pascal.
\end{itemize}
  
  \section*{Subtask}
  \begin{itemize}
    \item \textbf{Subtask 1 [30 punti]:} il tuo programma deve risolvere correttamente il caso d'esempio del testo (non vi serve nemmeno leggere l'input).
    \item \textbf{Subtask 2 [30 punti]:} il numero contenuto nel file \texttt{input.txt} \`e $13$.
    \item \textbf{Subtask 3 [40 punti]:} nessuna restrizione (oltre quella sul valore assoluto espressa nella sezione di ``Assunzioni'' generali).
  \end{itemize}


  \section*{Note generali sul sistema di sottoposizione (con valutazione a feedback immediato) delle vostre soluzioni}

Al sistema di sottoposizione va sottomesso solo il file sorgente del vostro programma. Il nostro server compiler\`a tale sorgente avvalendosi del compilatore
suggerito dall'estensione del file da voi sottomesso.

\vspace{0.2cm}
\begin{tabular}{|l|l|l|}
\hline
  linguaggio adottato  & estensione file sottomesso & compilatore/interprete utilizzato dal server  \\ \hline
\hline               
  python  & .py  & python3 \\ \hline
  bash    & .sh  & bash \\ \hline
  c++     & .cpp & g++ \\ \hline
  C       & .c   & gcc \\ \hline
  Pascal  & .pas & fpc \\ \hline
\hline               
\end{tabular}

\vspace{0.2cm}
Consigliamo di testare la soluzione in locale prima di sottometterla.
Se riscontrate difformit\`a di comportamento tra quanto in locale a quanto sul server, le esatte opzioni di compilazione utilizzate dal server sul singolo problema sono riportate nella pagina del problema sul sito.
