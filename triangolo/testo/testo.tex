% Template per generare 

\documentclass[a4paper,11pt]{article}
\usepackage{lmodern}
\renewcommand*\familydefault{\sfdefault}
\usepackage[utf8]{inputenc}
\usepackage[T1]{fontenc}
\usepackage[italian]{babel}
\usepackage{indentfirst}
\usepackage{graphicx}
\usepackage{tikz}
\newcommand*\circled[1]{\tikz[baseline=(char.base)]{
		\node[shape=circle,draw,inner sep=2pt] (char) {#1};}}
% \usepackage[group-separator={\,}]{siunitx}
\usepackage[left=2cm, right=2cm, bottom=3cm]{geometry}
\frenchspacing

\newcommand{\num}[1]{#1}

% Macro varie...
\newcommand{\file}[1]{\texttt{#1}}
\renewcommand{\arraystretch}{1.3}
\newcommand{\esempio}[2]{
\noindent\begin{minipage}{\textwidth}
\begin{tabular}{|p{11cm}|p{5cm}|}
	\hline
	\textbf{File \file{input.txt}} & \textbf{File \file{output.txt}}\\
	\hline
	\tt \small #1 &
	\tt \small #2 \\
	\hline
\end{tabular}
\end{minipage}
}

% Dati del task
\newcommand{\gara}{IOI 1994}
\newcommand{\nome}{Il triangolo}
\newcommand{\nomebreve}{triangolo}

\begin{document}
% Intestazione
\noindent{\Large \gara}
\vspace{0.5cm}

\noindent{\Huge \textbf \nome~(\texttt{\nomebreve})}

% Descrizione del task
\section*{Descrizione del problema}
\begin{tabular}{ccccccccc}
&&&&7&&&&\\
&&&3&&8&&&\\
&&8&&1&&0&&\\
&2&&7&&4&&4&\\
4&&5&&2&&6&&5\\
\end{tabular}\\

Questo è un triangolo! 

Scrivete un programma che calcoli la più grande somma di numeri ottenibile seguendo un percorso che parta dalla cima del triangolo e termini da qualche parte sulla sua base. Ad ogni passo si può procedere diagonalmente in basso o a destra o a sinistra.

% Input
\section*{File di input}
Il programma deve leggere da un file di nome \file{input.txt}. La prima riga del file contiene un unico intero N, il numero di righe del triangolo. Le successive N righe contengono un numero crescente di interi $i$, la $n$-esima riga contiene $n$ interi separati da uno spazio.

% Output
\section*{File di output}
Il programma deve scrivere in un file di nome \file{output.txt} la somma massima ottenibile dal triangolo.

% Assunzioni
\section*{Assunzioni}

\begin{itemize}
\item $1 < N \le 100$
\item $0 \le i < 100$
\end{itemize}

% Subtasks
\section*{Subtask}
\begin{itemize}
\item \textbf{Subtask 1 [\phantom{1}5 punti]:} caso di esempio.
\item \textbf{Subtask 2 [10 punti]:} $N \le 5$.
\item \textbf{Subtask 3 [15 punti]:} $N \le 10$.
\item \textbf{Subtask 4 [30 punti]:} $N \le 50$.
\item \textbf{Subtask 5 [15 punti]:} nessuna limitazione specifica.
\end{itemize}

% Esempi
\section*{Esempio di input/output}
\esempio{
5

7

3 8

8 1 0

2 7 4 4

4 5 2 6 5
}{30}

\end{document}