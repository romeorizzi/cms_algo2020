% Template per generare 

\documentclass[a4paper,11pt]{article}
\usepackage{lmodern}
\renewcommand*\familydefault{\sfdefault}
\usepackage[utf8]{inputenc}
\usepackage[T1]{fontenc}
\usepackage[italian]{babel}
\usepackage{indentfirst}
\usepackage{graphicx}
\usepackage{tikz}
\newcommand*\circled[1]{\tikz[baseline=(char.base)]{
		\node[shape=circle,draw,inner sep=2pt] (char) {#1};}}
% \usepackage[group-separator={\,}]{siunitx}
\usepackage[left=2cm, right=2cm, bottom=3cm]{geometry}
\frenchspacing

\newcommand{\num}[1]{#1}

% Macro varie...
\newcommand{\file}[1]{\texttt{#1}}
\renewcommand{\arraystretch}{1.3}
\newcommand{\esempio}[2]{
\noindent\begin{minipage}{\textwidth}
\begin{tabular}{|p{11cm}|p{5cm}|}
	\hline
	\textbf{File \file{input.txt}} & \textbf{File \file{output.txt}}\\
	\hline
	\tt \small #1 &
	\tt \small #2 \\
	\hline
\end{tabular}
\end{minipage}
}

% Dati del task
\newcommand{\gara}{Selezione territoriale 2004}
\newcommand{\nome}{La dieta di Poldo}
\newcommand{\nomebreve}{poldo}

\begin{document}
% Intestazione
\noindent{\Large \gara}
\vspace{0.5cm}

\noindent{\Huge \textbf \nome~(\texttt{\nomebreve})}

% Descrizione del task
\section*{Descrizione del problema}
Il dottore ordina a Poldo di seguire una dieta. Ad ogni pasto non può mai mangiare un panino che abbia un peso maggiore o uguale a quello appena mangiato. Quando Poldo passeggia per la via del suo paese, da ogni ristorante esce un cameriere proponendo il menù del giorno. Ciascun menù è composto da una serie di panini, che verranno serviti in un ordine ben definito, e dal peso di ciascun panino. Poldo, per non violare la regola della sua dieta, una volta scelto un menù, può decidere di mangiare o rifiutare un panino; se lo rifiuta il cameriere gli servirà il successivo e quello rifiutato non gli sarà più servito.

Si deve scrivere un programma che permetta a Poldo, leggendo un menù, di capire qual è il numero massimo di panini che può mangiare per quel menù senza violare la regola della sua dieta.

Riassumendo, Poldo può mangiare un panino se e solo se soddisfa una delle due condizioni:
\begin{itemize}
\item Il panino è il primo che mangia in un determinato pasto;
\item Il panino non ha un peso maggiore o uguale all'ultimo panino che ha mangiato in un determinato pasto.
\end{itemize}

% Input
\section*{File di input}
Il programma deve leggere da un file di nome \file{input.txt}. Nella prima è presente un intero N, il numero di panini nel menu. Le successive N righe contengono il peso in grammi $p$ del panino che verrà servito. I panini vengono serviti nell'ordine presentato.

% Output
\section*{File di output}
Il programma deve scrivere in un file di nome \file{output.txt}. Deve venire stampato un unico intero, il numero massimo di panini che Poldo può mangiare.

% Assunzioni
\section*{Assunzioni}

\begin{itemize}
\item $1 \le N \le 10\,000$
\item $0 \le p < 10\,000$
\end{itemize}

% Subtasks
\section*{Subtask}
\begin{itemize}
\item \textbf{Subtask 1 [\phantom{1}5 punti]:} casi di esempio.
\item \textbf{Subtask 2 [30 punti]:} $N \le 100$.
\item \textbf{Subtask 3 [25 punti]:} $N \le 1000$.
\item \textbf{Subtask 4 [25 punti]:} $N \le 3000$.
\item \textbf{Subtask 5 [15 punti]:} nessuna limitazione specifica.
\end{itemize}

% Esempi
\section*{Esempio di input/output}
\esempio{
5

3

6

7

5

3
}{3}

\esempio{
8

0

9

8

5

1

8

4

7
}{4}

\section*{Note}
Nel primo esempio, Poldo può mangiare i panini 6, 5, 3. Nel secondo esempio Poldo può mangiare i panini 9, 8, 5, 4 rispettando la sua dieta.

\end{document}