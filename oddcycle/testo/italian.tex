\renewcommand{\nomebreve}{oddcycle}
\renewcommand{\titolo}{Odd Cycle Finding}

\introduzione{}

Ape Maya \`e rimasta intrappolata in un nodo della tela di Tecla.
Tecla si affretta ad afferrarla ma, quando giunge su quel nodo,
si accorge di non avere appetito, e dice ``BLEACH''. Va detto che l'appetito dei ragni e' molto particolare: ogni volta che percorrono un filamento della loro rete,
essi invertono lo stato del loro stomaco tra ``SLURP'' e ``BLEACH''.
Tecla deve quindi farsi un giretto nella rete sperando di tornare da Maya
in stato ``SLURP''. 
Aiuta Tecla ad individuare una passeggiata funzionale al buon appetito.



\sezionetesto{Dati di input}
  
Il file \verb'input.txt' contiene nella prima riga gli interi
$N$ ed $M$, il numero di nodi e di collegamenti della rete.
Seguono $M$ righe, una per ogni collegamento.
Ciascuna di queste righe contiene due interi separati da spazio: $u$, $v$, dove $u$ e $v$ sono i due nodi ai capi del collegamento (due interi diversi tra $1$ ed $N$).

\sezionetesto{Dati di output}
  
Il file \verb'output.txt' deve contenere due linee.
La prima linea contiene un intero $L$,
il numero di spostamenti che Tecla compir\`a nella sua passeggiata.
Inizialmente Tecla \`e posta sul nodo $1$.
La seconda riga contiene $L+1$ numeri separati da spazio.
Il primo e l'ultimo di questi numeri sono necessariamente $1$ (nodo di partenza e di arrivo), gli altri sono i nodi come visitati da Tecla nell'ordine,
e possono aversi ripetizioni.


  \section*{Assunzioni}
  \begin{itemize}  
    \item $ 1 \leq M, N \leq 30$
    \item si garantisce l'esistenza di una soluzione:
          Ape Maya \`e spacciata!
  \end{itemize}


% Esempi
\sezionetesto{Esempio di input/output}

\begin{example}
  \exmpfile{oddcycle.input0.txt}{oddcycle.output0.txt}%
  \exmpfile{oddcycle.input1.txt}{oddcycle.output1.txt}%
\end{example}



  \section*{Subtask}
  \begin{itemize}
    \item \textbf{Subtask 1 [0 punti]:} i due esempi del testo.
    \item \textbf{Subtask 2 [10 punti]:} esiste il collegamento tra ogni coppia di nodi.
    \item \textbf{Subtask 3 [20 punti]:} il nodo 1 \`e adiacente ad ogni altro nodo. Ed almeno un ulteriore collegamento \`e presente.
    \item \textbf{Subtask 4 [20 punti]:} ogni nodo ha grado 2.
    \item \textbf{Subtask 5 [50 punti]:} nessuna restrizione, $M,N \leq 30$.
  \end{itemize}
  


\end{document}
