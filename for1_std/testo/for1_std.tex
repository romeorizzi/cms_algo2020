\renewcommand{\nomebreve}{for1\_std}
\renewcommand{\titolo}{Primo esercizio sul {\tt for} - emulare il comando {\tt seq} della {\tt bash}}

\introduzione{}

Ricevete in input un numero naturale positivo $N$ e,
in output, dovete generare la sequenza:\\

  $1$  \ \ $2$  \ \ $3$  \ \ $4$  \ \ $\cdots$  \ \ $N-2$  \ \ $N-1$  \ \ $N$

dei primi $N$ numeri naturali positivi, in bell'ordine.
Potete separare gli $N$ numeri in output tramite spazi oppure, se preferite, anche andando a capo emulando cos\'\i\ il comando \verb'seq' della \verb'bash' shell.
Chi conosce python penser\`a invece a:

\verb'print range(1,N+1)'

Ma, qualunque sia il linguaggio scelto, cercate di ottenere il risultato avvalendovi di un ciclo for.

\sezionetesto{Dati di input}
In ingresso riceverete solamente un numero intero e positivo $N$.

\sezionetesto{Dati di output}
In uscita dovete mandare, su una sola riga e separati da spazi, i primi $N$ numeri naturali positivi, in bell'ordine.
In altre parole, stampa su \verb'stdout' gli $N$ numeri della sequenza:\\

  $1$  \ \ $2$  \ \ $3$  \ \ $4$  \ \ $\cdots$  \ \ $N-2$  \ \ $N-1$  \ \ $N$


% Esempi
\sezionetesto{Esempio di input/output}
\esempio{6}{1 2 3 4 5 6}
\esempio{10}{1 2 3 4 5 6 7 8 9 10}

% Assunzioni
\sezionetesto{Assunzioni e note}
\begin{itemize}[nolistsep, noitemsep]
  \item $1 \le N \le 10\,000$.
\end{itemize}
  
  \section*{Subtask}
  \begin{itemize}
    \item \textbf{Subtask 0 [10 punti]:} i due esempi del testo.
    \item \textbf{Subtask 1 [15 punti]:} $N = 17$.
    \item \textbf{Subtask 2 [25 punti]:} $N \leq 20$.
    \item \textbf{Subtask 4 [50 punti]:} nessuna restrizione.
  \end{itemize}
  
