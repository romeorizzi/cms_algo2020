% Template per generare 

\documentclass[a4paper,11pt]{article}
\usepackage{lmodern}
\renewcommand*\familydefault{\sfdefault}
\usepackage[utf8]{inputenc}
\usepackage[T1]{fontenc}
\usepackage[italian]{babel}
\usepackage{indentfirst}
\usepackage{graphicx}
\usepackage{tikz}
\newcommand*\circled[1]{\tikz[baseline=(char.base)]{
		\node[shape=circle,draw,inner sep=2pt] (char) {#1};}}
% \usepackage[group-separator={\,}]{siunitx}
\usepackage[left=2cm, right=2cm, bottom=3cm]{geometry}
\frenchspacing

\newcommand{\num}[1]{#1}

% Macro varie...
\newcommand{\file}[1]{\texttt{#1}}
\renewcommand{\arraystretch}{1.3}
\newcommand{\esempio}[2]{
\noindent\begin{minipage}{\textwidth}
\begin{tabular}{|p{11cm}|p{5cm}|}
	\hline
	\textbf{File \file{input.txt}} & \textbf{File \file{output.txt}}\\
	\hline
	\tt \small #1 &
	\tt \small #2 \\
	\hline
\end{tabular}
\end{minipage}
}

% Dati del task
\newcommand{\gara}{phd2015}
\newcommand{\nome}{Campo minato}
\newcommand{\nomebreve}{minato}

\begin{document}
% Intestazione
\noindent{\Large \gara}
\vspace{0.5cm}

\noindent{\Huge \textbf \nome~(\texttt{\nomebreve})}

% Descrizione del task
\section*{Descrizione del problema}
Topolino \`e in missione per accompagnare una spedizione archeologica che segue un'antica mappa
acquisita di recente dal museo di Toponia. Raggiunta la localit\`a dove dovrebbe trovarsi un
prezioso e raro reperto archeologico, Topolino si imbatte in un labirinto che ha la forma di una
gigantesca scacchiera rettangolare composta di $N \times M$ lastroni di marmo di forma quadrata.

Nella mappa, le righe sono numerate da $1$ a $N$ e le colonne da $1$ a $M$. Il lastrone che si
trova nella posizione corrispondente alla riga $r$ e alla colonna $c$ viene identificato mediante la
coppia di interi $(r, c)$. I lastroni segnalati da una crocetta '+' sulla mappa contengono un
trabocchetto mortale e sono quindi da evitare, mentre i rimanenti sono innocui e segnalati da un
asterisco '*'.

Topolino deve partire dal lastrone in posizione $(1, 1)$ e raggiungere il lastrone in posizione $(N, M)$,
entrambi innocui. Può passare da un lastrone a un altro soltanto se questi condividono un lato, ossia muovendosi di un passo in direzione orizzontale oppure verticale, senza mai saltare. Ovviamente, tutti i lastroni visitati devono essere innocui.

Tuttavia, le insidie non sono finite qui: per poter attraversare incolume il labirinto, Topolino può
procedere solo spostandosi verso destra o verso il basso. Aiutalo a trovare il numero di possibili 
percorsi che può seguire.

% Input
\section*{File di input}
Il programma deve leggere da un file di nome \file{input.txt}. La prima riga contiene due interi positivi $N$ e $M$, separati da spazio, i quali rappresentano le dimensioni della scacchiera.
Le successive $N$ righe rappresentano il labirinto a scacchiera: la $r$-esima di tali righe contiene una
sequenza di $M$ caratteri '+' oppure '*', dove '+' indica un lastrone con trabocchetto mentre '*' indica
un lastrone sicuro. Tale riga rappresenta quindi i lastroni che si trovano sulla $r$-esima riga della
scacchiera: di conseguenza, il $c$-esimo carattere corrisponde al lastrone in posizione $(r, c)$. I caratteri NON sono separati da degli spazi.

% Output
\section*{File di output}
Il programma deve scrivere in un file di nome \file{output.txt}. Deve venire stampato un unico numero, il numero di percorsi che Topolino può seguire per arrivare dalla cella $(1,1)$ alla cella $(N,M)$.

% Assunzioni
\section*{Assunzioni}
\begin{itemize}
\item $0 < N,M \le 100$
\item $0 < r,c \le N$
\item È sempre possibile attraversare il labirinto dal lastrone in posizione $(1, 1)$ al lastrone in
posizione $(N, M)$, inoltre, tali due lastroni sono innocui.
\item Il risultato è interno all'intervallo rappresentato dagli int a $32$ bit.

\end{itemize}

% Subtasks
\section*{Subtask}
\begin{itemize}
\item \textbf{Subtask 1 [\phantom{1}5 punti]:} caso di esempio.
\item \textbf{Subtask 2 [30 punti]:} $N \le 20$.
\item \textbf{Subtask 3 [30 punti]:} $N \le 50$.
\item \textbf{Subtask 4 [35 punti]:} nessuna limitazione specifica.
\end{itemize}


% Esempi
\section*{Esempio di input/output}
\esempio{
5 4

****

+***

*+**

++**

++**
}{9}

\end{document}
