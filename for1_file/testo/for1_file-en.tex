\renewcommand{\nomebreve}{for1\_file}
\renewcommand{\titolo}{First exercise on {\tt for} - emulate the {\tt seq} command}

\introduzione{}

You receive in input a positive natural $N$ and,
in output, you must yield the sorted sequence of the first $N$ natural numbers:\\

  $1$  \ \ $2$  \ \ $3$  \ \ $4$  \ \ $\cdots$  \ \ $N-2$  \ \ $N-1$  \ \ $N$

You can separate the $N$ output numbers with spaces or, if you prefer,
with new lines, so that each number will go on a different line and you will be perfectly emulating the \verb'seq' command of the \verb'bash' shell.
Those of you who have seen some python will rather think of:

\verb'print range(1,N+1)'


\sezionetesto{input description}
The first and only line of the file \verb'input.txt'
contains a positive integer number $N$.

\sezionetesto{output description}
The first and only line of the file \verb'output.txt'
contains the sequence\\

  $1$  \ \ $2$  \ \ $3$  \ \ $4$  \ \ $\cdots$  \ \ $N-2$  \ \ $N-1$  \ \ $N$

comprising the first $N$ positive natural number in their order.
However, if you prefer,
you can yield an output file of $N$ lines,
with line~$i$ containing only the integer~$i$, 
possibly surrounded by spaces at your will.


% Esempi
\sezionetesto{input/output example}
\esempio{6}{1 2 3 4 5 6}
\esempio{10}{1 2 3 4 5 6 7 8 9 10}

% Assunzioni
\sezionetesto{Assumptions}
\begin{itemize}[nolistsep, noitemsep]
  \item $1 \le N \le 10\,000$.
\end{itemize}
  
  \section*{Subtasks}
  \begin{itemize}
    \item \textbf{Subtask 0 [10 punti]:} the two examples above.
    \item \textbf{Subtask 1 [15 punti]:} $N = 17$.
    \item \textbf{Subtask 2 [25 punti]:} $N \leq 20$.
    \item \textbf{Subtask 4 [50 punti]:} no special restriction.
  \end{itemize}
  
