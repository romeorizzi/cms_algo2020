\renewcommand{\nomebreve}{bungeejumping}
\renewcommand{\titolo}{Jump Jump}

\introduzione{}

Un jumper deve portarsi dalla prima cella $v[0]$ di un vettore $v$ all'ultima cella $v[N-1]$, tramite una sequenza di salti.
I salti avvengono secondo le seguenti regole:
Quando il jumper di trova nella cella $i$-esima del vettore ($0\leq i \leq N-1$)
egli può portarsi nella cella $i\pm (v[i] - r)$ correndo un rischio di gravità $r\geq 0$. La compagnia di assicurazioni gli chiede un premio pari al massimo valore di rischio $r$ incorso da un salto lungo il percorso stabilito (classe di rischio).
Aiuta il jumper ad individuare un percorso che mantenga il rischio il più basso possibile.




\sezionetesto{Dati di input}
La prima riga del file \verb'input.txt' contiene un interi positivo $N$, la lunghezza del vettore.
La seconda riga del file contiene gli $N$ valori del vettore, nell'ordine, e separati da spazi.
Si vedano i due esempi.

\sezionetesto{Dati di output}
Nel file \verb'output.txt' si scriva un'unica riga contenente
un unico numero naturale: il minimo premio possibile da pagare per la polizza.\\


% Esempi
\sezionetesto{Esempio di input/output}
\esempio{
7

3 3 3 3 3 3 3
}{0}

\esempio{
8

3 3 3 3 3 3 3 3
}{1}


% Assunzioni
\sezionetesto{Assunzioni e note}
\begin{itemize}[nolistsep, noitemsep]
  \item $1 \le N \le 500$.
\end{itemize}
  
  \section*{Subtask}
  \begin{itemize}
    \item \textbf{Subtask 1 [0 punti]:} i due esempi del testo.
    \item \textbf{Subtask 2 [20 punti]:} $N \leq 10$.
    \item \textbf{Subtask 3 [40 punti]:} $v[i] \leq 100$ per ogni $i = 0,1,\ldots, N-1$.
    \item \textbf{Subtask 4 [40 punti]:} nessuna restrizione.
  \end{itemize}
  
